\documentclass[12pt]{article}
\usepackage[table]{xcolor}
\usepackage[shortlabels]{enumitem}
\usepackage{tabularx,xltabular}
\usepackage{graphicx}
\usepackage{hyperref}
\usepackage{verbatim}
\usepackage{geometry}
\usepackage{ulem}
\usepackage[official]{eurosym}
\usepackage{tikz}
\usetikzlibrary{arrows,backgrounds,calc,decorations.markings,patterns,3d}
\usepackage{pgfplots}
\pgfplotsset{compat = newest}
\usetikzlibrary{fit}
\newcommand\addvmargin[1]{
\usetikzlibrary{arrows}
\node[fit=(current bounding box),inner ysep=#1,inner xsep=0]{};}
\usepackage{cancel}
\usepackage{fontspec}
\usepackage{array}  
\geometry{a4paper, top=2cm, left=2cm, right=2cm, bottom=2cm, headsep=1cm}
\usepackage{tabu}
\usepackage{pst-node}
\usepackage{colortbl}
\usepackage{array}
\usepackage{german}
\setlength\parindent{0pt}
\newcolumntype{?}{!{\vrule width 1pt}}
\usepackage{makecell}
\renewcommand{\arraystretch}{2.5}
\usepackage{pbox}
\usepackage{amssymb}
\usepackage{amsmath}
\usepackage{booktabs}
\newcolumntype{L}[1]{>{\raggedright\let\newline\\\arraybackslash\hspace{0pt}}m{#1}}
\newcolumntype{C}[1]{>{\centering\let\newline\\\arraybackslash\hspace{0pt}}m{#1}}
\newcolumntype{R}[1]{>{\raggedleft\let\newline\\\arraybackslash\hspace{0pt}}m{#1}}
\begin{document}
\rightline{Datum: 28.04.2023}
\centerline{{\Large Tägliche Übungen}} 
\vspace{1cm}
\noindent \\


\begin{xltabular}{\textwidth}{|C{0.75cm}|X|C{0.75cm}|X|}
\arrayrulecolor{black}\hline
a)&$\end{aligned}
&
b)&$\end{aligned}
\\\hline
c)&$\end{aligned}
&
d)&$\end{aligned}
\\\hline
e)&$\end{aligned}
&
f)&$\end{aligned}
\\\hline
g)&$\end{aligned}
&
h)&$\end{aligned}
\\\hline
i)&$\end{aligned}
&
j)&$\end{aligned}
\\\hline
k)&$\end{aligned}
&
l)&$\end{aligned}
\\\hline
m)&$\end{aligned}
&
n)&$\end{aligned}
\\\hline
o)&$\end{aligned}
&
p)&$\end{aligned}
\\\hline
\end{xltabular}
\vspace{0.5cm}
\newpage
\rightline{Datum: 28.04.2023}
\centerline{{\large Lösungen Tägliche Übungen}} 
\vspace{0.5cm}

\begin{xltabular}{\textwidth}{|C{0.75cm}|X|C{0.75cm}|X|}
\arrayrulecolor{black}\hline
a)&$\begin{aligned}
\textcolor{red}{b=6} & \rightarrow\\
y&=4 \cdot \textcolor{red}{b} - 1\\
y&=4 \cdot \textcolor{red}{6} - 1\\
y&=23\\
\end{aligned}$
&
b)&$\begin{aligned}
\textcolor{red}{x=10} & \rightarrow\\
y&=4 \cdot \textcolor{red}{x} - 5 \cdot \textcolor{red}{x}\\
y&=4 \cdot \textcolor{red}{10} - 5 \cdot \textcolor{red}{10}\\
y&=-10\\
\end{aligned}$
\\\hline
c)&$\begin{aligned}
\textcolor{red}{b=1} & \rightarrow\\
y&=2 \cdot \textcolor{red}{b} + 3\\
y&=2 \cdot \textcolor{red}{1} + 3\\
y&=5\\
\end{aligned}$
&
d)&$\begin{aligned}
\textcolor{red}{x=-7} & \rightarrow\\
y&=3 - \textcolor{red}{x}\\
y&=3 - \textcolor{red}{(-7)}\\
y&=10\\
\end{aligned}$
\\\hline
e)&$\begin{aligned}
\textcolor{red}{b=-2} & \rightarrow\\
y&=5 \cdot \textcolor{red}{b} + 4\\
y&=5 \cdot \textcolor{red}{(-2)} + 4\\
y&=-6\\
\end{aligned}$
&
f)&$\begin{aligned}
\textcolor{red}{a=12} & \rightarrow\\
y&=3 \cdot \textcolor{red}{a} + 2\\
y&=3 \cdot \textcolor{red}{12} + 2\\
y&=38\\
\end{aligned}$
\\\hline
g)&$\begin{aligned}
\textcolor{red}{b=3} & \rightarrow\\
y&=5 \cdot \textcolor{red}{b} + 2 \cdot \textcolor{red}{b}\\
y&=5 \cdot \textcolor{red}{3} + 2 \cdot \textcolor{red}{3}\\
y&=21\\
\end{aligned}$
&
h)&$\begin{aligned}
\textcolor{red}{z=-4} & \rightarrow\\
y&=4 \cdot \textcolor{red}{z} - 4 \cdot \textcolor{red}{z}\\
y&=4 \cdot \textcolor{red}{(-4)} - 4 \cdot \textcolor{red}{(-4)}\\
y&=0\\
\end{aligned}$
\\\hline
i)&$\begin{aligned}
\textcolor{red}{z=11} & \rightarrow\\
y&=3 + 3 \cdot \textcolor{red}{z}\\
y&=3 + 3 \cdot \textcolor{red}{11}\\
y&=36\\
\end{aligned}$
&
j)&$\begin{aligned}
\textcolor{red}{b=6} & \rightarrow\\
y&=5 \cdot \textcolor{red}{b} + 5\\
y&=5 \cdot \textcolor{red}{6} + 5\\
y&=35\\
\end{aligned}$
\\\hline
k)&$\begin{aligned}
\textcolor{red}{x=-7} & \rightarrow\\
y&=2 + 3 \cdot \textcolor{red}{x}\\
y&=2 + 3 \cdot \textcolor{red}{(-7)}\\
y&=-19\\
\end{aligned}$
&
l)&$\begin{aligned}
\textcolor{red}{b=12} & \rightarrow\\
y&=2 \cdot \textcolor{red}{b} - 3\\
y&=2 \cdot \textcolor{red}{12} - 3\\
y&=21\\
\end{aligned}$
\\\hline
m)&$\begin{aligned}
\textcolor{red}{z=1} & \rightarrow\\
y&=4 + 3 \cdot \textcolor{red}{z}\\
y&=4 + 3 \cdot \textcolor{red}{1}\\
y&=7\\
\end{aligned}$
&
n)&$\begin{aligned}
\textcolor{red}{a=-2} & \rightarrow\\
y&=\textcolor{red}{a} + \textcolor{red}{a}\\
y&=\textcolor{red}{(-2)} + \textcolor{red}{(-2)}\\
y&=-4\\
\end{aligned}$
\\\hline
o)&$\begin{aligned}
\textcolor{red}{x=8} & \rightarrow\\
y&=2 \cdot \textcolor{red}{x} + 5\\
y&=2 \cdot \textcolor{red}{8} + 5\\
y&=21\\
\end{aligned}$
&
p)&$\begin{aligned}
\textcolor{red}{b=-3} & \rightarrow\\
y&=3 \cdot \textcolor{red}{b} + 4\\
y&=3 \cdot \textcolor{red}{(-3)} + 4\\
y&=-5\\
\end{aligned}$
\\\hline
\end{xltabular}
\vspace{0.5cm}
\end{document}