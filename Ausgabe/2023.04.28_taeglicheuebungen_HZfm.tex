\documentclass[12pt]{article}
\usepackage[table]{xcolor}
\usepackage[shortlabels]{enumitem}
\usepackage{tabularx,xltabular}
\usepackage{graphicx}
\usepackage{hyperref}
\usepackage{verbatim}
\usepackage{geometry}
\usepackage{ulem}
\usepackage[official]{eurosym}
\usepackage{tikz}
\usetikzlibrary{arrows,backgrounds,calc,decorations.markings,patterns,3d}
\usepackage{pgfplots}
\pgfplotsset{compat = newest}
\usetikzlibrary{fit}
\newcommand\addvmargin[1]{
\usetikzlibrary{arrows}
\node[fit=(current bounding box),inner ysep=#1,inner xsep=0]{};}
\usepackage{cancel}
\usepackage{fontspec}
\usepackage{array}  
\geometry{a4paper, top=2cm, left=2cm, right=2cm, bottom=2cm, headsep=1cm}
\usepackage{tabu}
\usepackage{pst-node}
\usepackage{colortbl}
\usepackage{array}
\usepackage{german}
\setlength\parindent{0pt}
\newcolumntype{?}{!{\vrule width 1pt}}
\usepackage{makecell}
\renewcommand{\arraystretch}{2.5}
\usepackage{pbox}
\usepackage{amssymb}
\usepackage{amsmath}
\usepackage{booktabs}
\newcolumntype{L}[1]{>{\raggedright\let\newline\\\arraybackslash\hspace{0pt}}m{#1}}
\newcolumntype{C}[1]{>{\centering\let\newline\\\arraybackslash\hspace{0pt}}m{#1}}
\newcolumntype{R}[1]{>{\raggedleft\let\newline\\\arraybackslash\hspace{0pt}}m{#1}}
\begin{document}
\rightline{Datum: 28.04.2023}
\centerline{{\Large Tägliche Übungen}} 
\vspace{1cm}
\noindent \\


\begin{xltabular}{\textwidth}{|C{0.75cm}|X|C{0.75cm}|X|}
\arrayrulecolor{black}\hline
a)&Setze für die Variabel b den Wert -10 ein und berechne die Lösung für y:$$y=2 \cdot b + 3$$
&
b)&Setze für die Variabel z den Wert 9 ein und berechne die Lösung für y:$$y=4 \cdot z + 5 \cdot z$$
\\\hline
c)&Setze für die Variabel x den Wert -2 ein und berechne die Lösung für y:$$y=5 \cdot x + 1$$
&
d)&Setze für die Variabel z den Wert -11 ein und berechne die Lösung für y:$$y=5 \cdot z + 3$$
\\\hline
e)&Setze für die Variabel a den Wert 11 ein und berechne die Lösung für y:$$y=2 \cdot a - 5 \cdot a$$
&
f)&Setze für die Variabel z den Wert 6 ein und berechne die Lösung für y:$$y=z + 1$$
\\\hline
g)&Setze für die Variabel x den Wert -11 ein und berechne die Lösung für y:$$y=2 + 4 \cdot x$$
&
h)&Setze für die Variabel b den Wert -9 ein und berechne die Lösung für y:$$y=5 - 3 \cdot b$$
\\\hline
i)&Setze für die Variabel z den Wert 4 ein und berechne die Lösung für y:$$y=4 \cdot z + 4$$
&
j)&Setze für die Variabel a den Wert -7 ein und berechne die Lösung für y:$$y=2 \cdot a + a$$
\\\hline
k)&Setze für die Variabel z den Wert -1 ein und berechne die Lösung für y:$$y=2 \cdot z + z$$
&
l)&Setze für die Variabel z den Wert -1 ein und berechne die Lösung für y:$$y=2 \cdot z - 5 \cdot z$$
\\\hline
m)&Setze für die Variabel z den Wert 5 ein und berechne die Lösung für y:$$y=z - 1$$
&
n)&Setze für die Variabel z den Wert -6 ein und berechne die Lösung für y:$$y=1 - z$$
\\\hline
o)&Setze für die Variabel a den Wert -11 ein und berechne die Lösung für y:$$y=5 - 2 \cdot a$$
&
p)&Setze für die Variabel z den Wert 9 ein und berechne die Lösung für y:$$y=3 - 3 \cdot z$$
\\\hline
q)&Setze für die Variabel z den Wert -7 ein und berechne die Lösung für y:$$y=4 \cdot z - 1$$
&
r)&Setze für die Variabel a den Wert -2 ein und berechne die Lösung für y:$$y=5 \cdot a + 5$$
\\\hline
\end{xltabular}
\vspace{0.5cm}
\newpage
\rightline{Datum: 28.04.2023}
\centerline{{\large Lösungen Tägliche Übungen}} 
\vspace{0.5cm}

\begin{xltabular}{\textwidth}{|C{0.75cm}|X|C{0.75cm}|X|}
\arrayrulecolor{black}\hline
a)&$\begin{aligned}
\textcolor{red}{b=-10} & \rightarrow
y&=2 \cdot \textcolor{red}{b} + 3\\
y&=2 \cdot \textcolor{red}{(-10)} + 3\\
y&=-17\\
\end{aligned}$
&
b)&$\begin{aligned}
\textcolor{red}{z=9} & \rightarrow
y&=4 \cdot \textcolor{red}{z} + 5 \cdot \textcolor{red}{z}\\
y&=4 \cdot \textcolor{red}{9} + 5 \cdot \textcolor{red}{9}\\
y&=81\\
\end{aligned}$
\\\hline
c)&$\begin{aligned}
\textcolor{red}{x=-2} & \rightarrow
y&=5 \cdot \textcolor{red}{x} + 1\\
y&=5 \cdot \textcolor{red}{(-2)} + 1\\
y&=-9\\
\end{aligned}$
&
d)&$\begin{aligned}
\textcolor{red}{z=-11} & \rightarrow
y&=5 \cdot \textcolor{red}{z} + 3\\
y&=5 \cdot \textcolor{red}{(-11)} + 3\\
y&=-52\\
\end{aligned}$
\\\hline
e)&$\begin{aligned}
\textcolor{red}{a=11} & \rightarrow
y&=2 \cdot \textcolor{red}{a} - 5 \cdot \textcolor{red}{a}\\
y&=2 \cdot \textcolor{red}{11} - 5 \cdot \textcolor{red}{11}\\
y&=-33\\
\end{aligned}$
&
f)&$\begin{aligned}
\textcolor{red}{z=6} & \rightarrow
y&=\textcolor{red}{z} + 1\\
y&=\textcolor{red}{6} + 1\\
y&=7\\
\end{aligned}$
\\\hline
g)&$\begin{aligned}
\textcolor{red}{x=-11} & \rightarrow
y&=2 + 4 \cdot \textcolor{red}{x}\\
y&=2 + 4 \cdot \textcolor{red}{(-11)}\\
y&=-42\\
\end{aligned}$
&
h)&$\begin{aligned}
\textcolor{red}{b=-9} & \rightarrow
y&=5 - 3 \cdot \textcolor{red}{b}\\
y&=5 - 3 \cdot \textcolor{red}{(-9)}\\
y&=32\\
\end{aligned}$
\\\hline
i)&$\begin{aligned}
\textcolor{red}{z=4} & \rightarrow
y&=4 \cdot \textcolor{red}{z} + 4\\
y&=4 \cdot \textcolor{red}{4} + 4\\
y&=20\\
\end{aligned}$
&
j)&$\begin{aligned}
\textcolor{red}{a=-7} & \rightarrow
y&=2 \cdot \textcolor{red}{a} + \textcolor{red}{a}\\
y&=2 \cdot \textcolor{red}{(-7)} + \textcolor{red}{(-7)}\\
y&=-21\\
\end{aligned}$
\\\hline
k)&$\begin{aligned}
\textcolor{red}{z=-1} & \rightarrow
y&=2 \cdot \textcolor{red}{z} + \textcolor{red}{z}\\
y&=2 \cdot \textcolor{red}{(-1)} + \textcolor{red}{(-1)}\\
y&=-3\\
\end{aligned}$
&
l)&$\begin{aligned}
\textcolor{red}{z=-1} & \rightarrow
y&=2 \cdot \textcolor{red}{z} - 5 \cdot \textcolor{red}{z}\\
y&=2 \cdot \textcolor{red}{(-1)} - 5 \cdot \textcolor{red}{(-1)}\\
y&=3\\
\end{aligned}$
\\\hline
m)&$\begin{aligned}
\textcolor{red}{z=5} & \rightarrow
y&=\textcolor{red}{z} - 1\\
y&=\textcolor{red}{5} - 1\\
y&=4\\
\end{aligned}$
&
n)&$\begin{aligned}
\textcolor{red}{z=-6} & \rightarrow
y&=1 - \textcolor{red}{z}\\
y&=1 - \textcolor{red}{(-6)}\\
y&=7\\
\end{aligned}$
\\\hline
o)&$\begin{aligned}
\textcolor{red}{a=-11} & \rightarrow
y&=5 - 2 \cdot \textcolor{red}{a}\\
y&=5 - 2 \cdot \textcolor{red}{(-11)}\\
y&=27\\
\end{aligned}$
&
p)&$\begin{aligned}
\textcolor{red}{z=9} & \rightarrow
y&=3 - 3 \cdot \textcolor{red}{z}\\
y&=3 - 3 \cdot \textcolor{red}{9}\\
y&=-24\\
\end{aligned}$
\\\hline
q)&$\begin{aligned}
\textcolor{red}{z=-7} & \rightarrow
y&=4 \cdot \textcolor{red}{z} - 1\\
y&=4 \cdot \textcolor{red}{(-7)} - 1\\
y&=-29\\
\end{aligned}$
&
r)&$\begin{aligned}
\textcolor{red}{a=-2} & \rightarrow
y&=5 \cdot \textcolor{red}{a} + 5\\
y&=5 \cdot \textcolor{red}{(-2)} + 5\\
y&=-5\\
\end{aligned}$
\\\hline
\end{xltabular}
\vspace{0.5cm}
\end{document}