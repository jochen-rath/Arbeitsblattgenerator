\documentclass[12pt]{article}
\usepackage[table]{xcolor}
\usepackage[shortlabels]{enumitem}
\usepackage{tabularx,xltabular}
\usepackage{graphicx}
\usepackage{hyperref}
\usepackage{verbatim}
\usepackage{geometry}
\usepackage{ulem}
\usepackage[official]{eurosym}
\usepackage{tikz}
\usetikzlibrary{arrows,backgrounds,calc,decorations.markings,patterns,3d}
\usepackage{pgfplots}
\pgfplotsset{compat = newest}
\usetikzlibrary{fit}
\newcommand\addvmargin[1]{
\usetikzlibrary{arrows}
\node[fit=(current bounding box),inner ysep=#1,inner xsep=0]{};}
\usepackage{cancel}
\usepackage{fontspec}
\usepackage{array}  
\geometry{a4paper, top=2cm, left=2cm, right=2cm, bottom=2cm, headsep=1cm}
\usepackage{tabu}
\usepackage{pst-node}
\usepackage{colortbl}
\usepackage{array}
\usepackage{german}
\setlength\parindent{0pt}
\newcolumntype{?}{!{\vrule width 1pt}}
\usepackage{makecell}
\renewcommand{\arraystretch}{2.5}
\usepackage{pbox}
\usepackage{amssymb}
\usepackage{amsmath}
\usepackage{booktabs}
\newcolumntype{L}[1]{>{\raggedright\let\newline\\\arraybackslash\hspace{0pt}}m{#1}}
\newcolumntype{C}[1]{>{\centering\let\newline\\\arraybackslash\hspace{0pt}}m{#1}}
\newcolumntype{R}[1]{>{\raggedleft\let\newline\\\arraybackslash\hspace{0pt}}m{#1}}
\begin{document}
\rightline{Datum: 28.04.2023}
\centerline{{\Large Tägliche Übungen}} 
\vspace{1cm}
\noindent \\


\begin{xltabular}{\textwidth}{|C{0.75cm}|X|C{0.75cm}|X|}
\arrayrulecolor{black}\hline
a)&Setze für die Variabel b den Wert 1 ein und berechne die Lösung für y:$$y=3 \cdot b - 4 \cdot b$$
&
b)&Setze für die Variabel b den Wert 12 ein und berechne die Lösung für y:$$y=1 + 3 \cdot b$$
\\\hline
c)&Setze für die Variabel z den Wert -3 ein und berechne die Lösung für y:$$y=5 - z$$
&
d)&Setze für die Variabel b den Wert -9 ein und berechne die Lösung für y:$$y=5 \cdot b + b$$
\\\hline
e)&Setze für die Variabel z den Wert 4 ein und berechne die Lösung für y:$$y=1 - 2 \cdot z$$
&
f)&Setze für die Variabel a den Wert 7 ein und berechne die Lösung für y:$$y=3 \cdot a - 4 \cdot a$$
\\\hline
g)&Setze für die Variabel b den Wert -9 ein und berechne die Lösung für y:$$y=b - 5 \cdot b$$
&
h)&Setze für die Variabel x den Wert -11 ein und berechne die Lösung für y:$$y=3 \cdot x - 4$$
\\\hline
i)&Setze für die Variabel b den Wert 11 ein und berechne die Lösung für y:$$y=3 \cdot b + 3 \cdot b$$
&
j)&Setze für die Variabel a den Wert 11 ein und berechne die Lösung für y:$$y=5 \cdot a - 5 \cdot a$$
\\\hline
k)&Setze für die Variabel a den Wert 2 ein und berechne die Lösung für y:$$y=4 + 5 \cdot a$$
&
l)&Setze für die Variabel b den Wert 3 ein und berechne die Lösung für y:$$y=5 \cdot b - 2$$
\\\hline
m)&Setze für die Variabel b den Wert -8 ein und berechne die Lösung für y:$$y=4 \cdot b + 5$$
&
n)&Setze für die Variabel b den Wert 10 ein und berechne die Lösung für y:$$y=2 \cdot b - 5 \cdot b$$
\\\hline
o)&Setze für die Variabel z den Wert 4 ein und berechne die Lösung für y:$$y=2 + 4 \cdot z$$
&
p)&Setze für die Variabel x den Wert 1 ein und berechne die Lösung für y:$$y=1 - 4 \cdot x$$
\\\hline
q)&Setze für die Variabel b den Wert 6 ein und berechne die Lösung für y:$$y=3 \cdot b - 2 \cdot b$$
&
r)&Setze für die Variabel z den Wert 1 ein und berechne die Lösung für y:$$y=z + 4$$
\\\hline
\end{xltabular}
\vspace{0.5cm}
\newpage
\rightline{Datum: 28.04.2023}
\centerline{{\large Lösungen Tägliche Übungen}} 
\vspace{0.5cm}

\begin{xltabular}{\textwidth}{|C{0.75cm}|X|C{0.75cm}|X|}
\arrayrulecolor{black}\hline
a)&$\begin{aligned}
\textcolor{red}{b=1} & \rightarrow\\
y&=3 \cdot \textcolor{red}{b} - 4 \cdot \textcolor{red}{b}\\
y&=3 \cdot \textcolor{red}{1} - 4 \cdot \textcolor{red}{1}\\
y&=-1\\
\end{aligned}$
&
b)&$\begin{aligned}
\textcolor{red}{b=12} & \rightarrow\\
y&=1 + 3 \cdot \textcolor{red}{b}\\
y&=1 + 3 \cdot \textcolor{red}{12}\\
y&=37\\
\end{aligned}$
\\\hline
c)&$\begin{aligned}
\textcolor{red}{z=-3} & \rightarrow\\
y&=5 - \textcolor{red}{z}\\
y&=5 - \textcolor{red}{(-3)}\\
y&=8\\
\end{aligned}$
&
d)&$\begin{aligned}
\textcolor{red}{b=-9} & \rightarrow\\
y&=5 \cdot \textcolor{red}{b} + \textcolor{red}{b}\\
y&=5 \cdot \textcolor{red}{(-9)} + \textcolor{red}{(-9)}\\
y&=-54\\
\end{aligned}$
\\\hline
e)&$\begin{aligned}
\textcolor{red}{z=4} & \rightarrow\\
y&=1 - 2 \cdot \textcolor{red}{z}\\
y&=1 - 2 \cdot \textcolor{red}{4}\\
y&=-7\\
\end{aligned}$
&
f)&$\begin{aligned}
\textcolor{red}{a=7} & \rightarrow\\
y&=3 \cdot \textcolor{red}{a} - 4 \cdot \textcolor{red}{a}\\
y&=3 \cdot \textcolor{red}{7} - 4 \cdot \textcolor{red}{7}\\
y&=-7\\
\end{aligned}$
\\\hline
g)&$\begin{aligned}
\textcolor{red}{b=-9} & \rightarrow\\
y&=\textcolor{red}{b} - 5 \cdot \textcolor{red}{b}\\
y&=\textcolor{red}{(-9)} - 5 \cdot \textcolor{red}{(-9)}\\
y&=36\\
\end{aligned}$
&
h)&$\begin{aligned}
\textcolor{red}{x=-11} & \rightarrow\\
y&=3 \cdot \textcolor{red}{x} - 4\\
y&=3 \cdot \textcolor{red}{(-11)} - 4\\
y&=-37\\
\end{aligned}$
\\\hline
i)&$\begin{aligned}
\textcolor{red}{b=11} & \rightarrow\\
y&=3 \cdot \textcolor{red}{b} + 3 \cdot \textcolor{red}{b}\\
y&=3 \cdot \textcolor{red}{11} + 3 \cdot \textcolor{red}{11}\\
y&=66\\
\end{aligned}$
&
j)&$\begin{aligned}
\textcolor{red}{a=11} & \rightarrow\\
y&=5 \cdot \textcolor{red}{a} - 5 \cdot \textcolor{red}{a}\\
y&=5 \cdot \textcolor{red}{11} - 5 \cdot \textcolor{red}{11}\\
y&=0\\
\end{aligned}$
\\\hline
k)&$\begin{aligned}
\textcolor{red}{a=2} & \rightarrow\\
y&=4 + 5 \cdot \textcolor{red}{a}\\
y&=4 + 5 \cdot \textcolor{red}{2}\\
y&=14\\
\end{aligned}$
&
l)&$\begin{aligned}
\textcolor{red}{b=3} & \rightarrow\\
y&=5 \cdot \textcolor{red}{b} - 2\\
y&=5 \cdot \textcolor{red}{3} - 2\\
y&=13\\
\end{aligned}$
\\\hline
m)&$\begin{aligned}
\textcolor{red}{b=-8} & \rightarrow\\
y&=4 \cdot \textcolor{red}{b} + 5\\
y&=4 \cdot \textcolor{red}{(-8)} + 5\\
y&=-27\\
\end{aligned}$
&
n)&$\begin{aligned}
\textcolor{red}{b=10} & \rightarrow\\
y&=2 \cdot \textcolor{red}{b} - 5 \cdot \textcolor{red}{b}\\
y&=2 \cdot \textcolor{red}{10} - 5 \cdot \textcolor{red}{10}\\
y&=-30\\
\end{aligned}$
\\\hline
o)&$\begin{aligned}
\textcolor{red}{z=4} & \rightarrow\\
y&=2 + 4 \cdot \textcolor{red}{z}\\
y&=2 + 4 \cdot \textcolor{red}{4}\\
y&=18\\
\end{aligned}$
&
p)&$\begin{aligned}
\textcolor{red}{x=1} & \rightarrow\\
y&=1 - 4 \cdot \textcolor{red}{x}\\
y&=1 - 4 \cdot \textcolor{red}{1}\\
y&=-3\\
\end{aligned}$
\\\hline
q)&$\begin{aligned}
\textcolor{red}{b=6} & \rightarrow\\
y&=3 \cdot \textcolor{red}{b} - 2 \cdot \textcolor{red}{b}\\
y&=3 \cdot \textcolor{red}{6} - 2 \cdot \textcolor{red}{6}\\
y&=6\\
\end{aligned}$
&
r)&$\begin{aligned}
\textcolor{red}{z=1} & \rightarrow\\
y&=\textcolor{red}{z} + 4\\
y&=\textcolor{red}{1} + 4\\
y&=5\\
\end{aligned}$
\\\hline
\end{xltabular}
\vspace{0.5cm}
\end{document}