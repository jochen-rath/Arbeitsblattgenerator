\documentclass[12pt]{article}
\usepackage[table]{xcolor}
\usepackage[shortlabels]{enumitem}
\usepackage{tabularx,xltabular}
\usepackage{graphicx}
\usepackage{hyperref}
\usepackage{verbatim}
\usepackage{geometry}
\usepackage{ulem}
\usepackage[official]{eurosym}
\usepackage{tikz}
\usetikzlibrary{arrows,backgrounds,calc,decorations.markings,patterns,3d}
\usepackage{pgfplots}
\pgfplotsset{compat = newest}
\usetikzlibrary{fit}
\newcommand\addvmargin[1]{
\usetikzlibrary{arrows}
\node[fit=(current bounding box),inner ysep=#1,inner xsep=0]{};}
\usepackage{cancel}
\usepackage{fontspec}
\usepackage{array}  
\geometry{a4paper, top=2cm, left=2cm, right=2cm, bottom=2cm, headsep=1cm}
\usepackage{tabu}
\usepackage{pst-node}
\usepackage{colortbl}
\usepackage{array}
\usepackage{german}
\setlength\parindent{0pt}
\newcolumntype{?}{!{\vrule width 1pt}}
\usepackage{makecell}
\renewcommand{\arraystretch}{2.5}
\usepackage{pbox}
\usepackage{amssymb}
\usepackage{amsmath}
\usepackage{booktabs}
\newcolumntype{L}[1]{>{\raggedright\let\newline\\\arraybackslash\hspace{0pt}}m{#1}}
\newcolumntype{C}[1]{>{\centering\let\newline\\\arraybackslash\hspace{0pt}}m{#1}}
\newcolumntype{R}[1]{>{\raggedleft\let\newline\\\arraybackslash\hspace{0pt}}m{#1}}
\begin{document}
\rightline{Datum: 28.04.2023}
\centerline{{\Large Tägliche Übungen}} 
\vspace{1cm}
\noindent \\


\begin{xltabular}{\textwidth}{|C{0.75cm}|X|C{0.75cm}|X|}
\arrayrulecolor{black}\hline
a)&$\begin{aligned}
y=&4 + 3*a\\
 a=-3~ \rightarrow ~ y=?
\end{aligned}$
&
b)&$\begin{aligned}
y=&3*b - 2*b\\
 b=8~ \rightarrow ~ y=?
\end{aligned}$
\\\hline
c)&$\begin{aligned}
y=&2*x - 5*x\\
 x=-5~ \rightarrow ~ y=?
\end{aligned}$
&
d)&$\begin{aligned}
y=&4*a - 2*a\\
 a=7~ \rightarrow ~ y=?
\end{aligned}$
\\\hline
e)&$\begin{aligned}
y=&5 - b\\
 b=6~ \rightarrow ~ y=?
\end{aligned}$
&
f)&$\begin{aligned}
y=&2 - 5*a\\
 a=6~ \rightarrow ~ y=?
\end{aligned}$
\\\hline
g)&$\begin{aligned}
y=&3*b - b\\
 b=1~ \rightarrow ~ y=?
\end{aligned}$
&
h)&$\begin{aligned}
y=&4*x - 3*x\\
 x=5~ \rightarrow ~ y=?
\end{aligned}$
\\\hline
i)&$\begin{aligned}
y=&4 + 3*a\\
 a=3~ \rightarrow ~ y=?
\end{aligned}$
&
j)&$\begin{aligned}
y=&3 + 5*z\\
 z=-12~ \rightarrow ~ y=?
\end{aligned}$
\\\hline
k)&$\begin{aligned}
y=&5 - x\\
 x=-12~ \rightarrow ~ y=?
\end{aligned}$
&
l)&$\begin{aligned}
y=&3*x + 4*x\\
 x=-6~ \rightarrow ~ y=?
\end{aligned}$
\\\hline
m)&$\begin{aligned}
y=&2*a + 4*a\\
 a=-1~ \rightarrow ~ y=?
\end{aligned}$
&
n)&$\begin{aligned}
y=&2*z - 3\\
 z=8~ \rightarrow ~ y=?
\end{aligned}$
\\\hline
o)&$\begin{aligned}
y=&2 - z\\
 z=-11~ \rightarrow ~ y=?
\end{aligned}$
&
p)&$\begin{aligned}
y=&5 + 3*a\\
 a=-9~ \rightarrow ~ y=?
\end{aligned}$
\\\hline
q)&$\begin{aligned}
y=&2*a + a\\
 a=-4~ \rightarrow ~ y=?
\end{aligned}$
&
r)&$\begin{aligned}
y=&4 - 2*z\\
 z=8~ \rightarrow ~ y=?
\end{aligned}$
\\\hline
\end{xltabular}
\vspace{0.5cm}
\newpage
\rightline{Datum: 28.04.2023}
\centerline{{\large Lösungen Tägliche Übungen}} 
\vspace{0.5cm}

\begin{xltabular}{\textwidth}{|C{0.75cm}|X|C{0.75cm}|X|}
\arrayrulecolor{black}\hline
a)&$\begin{aligned}
\textcolor{red}{a=-3} & \rightarrow\\
y&=4 + 3 \cdot \textcolor{red}{a}\\
y&=4 + 3 \cdot \textcolor{red}{(-3)}\\
y&=-5\\
\end{aligned}$
&
b)&$\begin{aligned}
\textcolor{red}{b=8} & \rightarrow\\
y&=3 \cdot \textcolor{red}{b} - 2 \cdot \textcolor{red}{b}\\
y&=3 \cdot \textcolor{red}{8} - 2 \cdot \textcolor{red}{8}\\
y&=8\\
\end{aligned}$
\\\hline
c)&$\begin{aligned}
\textcolor{red}{x=-5} & \rightarrow\\
y&=2 \cdot \textcolor{red}{x} - 5 \cdot \textcolor{red}{x}\\
y&=2 \cdot \textcolor{red}{(-5)} - 5 \cdot \textcolor{red}{(-5)}\\
y&=15\\
\end{aligned}$
&
d)&$\begin{aligned}
\textcolor{red}{a=7} & \rightarrow\\
y&=4 \cdot \textcolor{red}{a} - 2 \cdot \textcolor{red}{a}\\
y&=4 \cdot \textcolor{red}{7} - 2 \cdot \textcolor{red}{7}\\
y&=14\\
\end{aligned}$
\\\hline
e)&$\begin{aligned}
\textcolor{red}{b=6} & \rightarrow\\
y&=5 - \textcolor{red}{b}\\
y&=5 - \textcolor{red}{6}\\
y&=-1\\
\end{aligned}$
&
f)&$\begin{aligned}
\textcolor{red}{a=6} & \rightarrow\\
y&=2 - 5 \cdot \textcolor{red}{a}\\
y&=2 - 5 \cdot \textcolor{red}{6}\\
y&=-28\\
\end{aligned}$
\\\hline
g)&$\begin{aligned}
\textcolor{red}{b=1} & \rightarrow\\
y&=3 \cdot \textcolor{red}{b} - \textcolor{red}{b}\\
y&=3 \cdot \textcolor{red}{1} - \textcolor{red}{1}\\
y&=2\\
\end{aligned}$
&
h)&$\begin{aligned}
\textcolor{red}{x=5} & \rightarrow\\
y&=4 \cdot \textcolor{red}{x} - 3 \cdot \textcolor{red}{x}\\
y&=4 \cdot \textcolor{red}{5} - 3 \cdot \textcolor{red}{5}\\
y&=5\\
\end{aligned}$
\\\hline
i)&$\begin{aligned}
\textcolor{red}{a=3} & \rightarrow\\
y&=4 + 3 \cdot \textcolor{red}{a}\\
y&=4 + 3 \cdot \textcolor{red}{3}\\
y&=13\\
\end{aligned}$
&
j)&$\begin{aligned}
\textcolor{red}{z=-12} & \rightarrow\\
y&=3 + 5 \cdot \textcolor{red}{z}\\
y&=3 + 5 \cdot \textcolor{red}{(-12)}\\
y&=-57\\
\end{aligned}$
\\\hline
k)&$\begin{aligned}
\textcolor{red}{x=-12} & \rightarrow\\
y&=5 - \textcolor{red}{x}\\
y&=5 - \textcolor{red}{(-12)}\\
y&=17\\
\end{aligned}$
&
l)&$\begin{aligned}
\textcolor{red}{x=-6} & \rightarrow\\
y&=3 \cdot \textcolor{red}{x} + 4 \cdot \textcolor{red}{x}\\
y&=3 \cdot \textcolor{red}{(-6)} + 4 \cdot \textcolor{red}{(-6)}\\
y&=-42\\
\end{aligned}$
\\\hline
m)&$\begin{aligned}
\textcolor{red}{a=-1} & \rightarrow\\
y&=2 \cdot \textcolor{red}{a} + 4 \cdot \textcolor{red}{a}\\
y&=2 \cdot \textcolor{red}{(-1)} + 4 \cdot \textcolor{red}{(-1)}\\
y&=-6\\
\end{aligned}$
&
n)&$\begin{aligned}
\textcolor{red}{z=8} & \rightarrow\\
y&=2 \cdot \textcolor{red}{z} - 3\\
y&=2 \cdot \textcolor{red}{8} - 3\\
y&=13\\
\end{aligned}$
\\\hline
o)&$\begin{aligned}
\textcolor{red}{z=-11} & \rightarrow\\
y&=2 - \textcolor{red}{z}\\
y&=2 - \textcolor{red}{(-11)}\\
y&=13\\
\end{aligned}$
&
p)&$\begin{aligned}
\textcolor{red}{a=-9} & \rightarrow\\
y&=5 + 3 \cdot \textcolor{red}{a}\\
y&=5 + 3 \cdot \textcolor{red}{(-9)}\\
y&=-22\\
\end{aligned}$
\\\hline
q)&$\begin{aligned}
\textcolor{red}{a=-4} & \rightarrow\\
y&=2 \cdot \textcolor{red}{a} + \textcolor{red}{a}\\
y&=2 \cdot \textcolor{red}{(-4)} + \textcolor{red}{(-4)}\\
y&=-12\\
\end{aligned}$
&
r)&$\begin{aligned}
\textcolor{red}{z=8} & \rightarrow\\
y&=4 - 2 \cdot \textcolor{red}{z}\\
y&=4 - 2 \cdot \textcolor{red}{8}\\
y&=-12\\
\end{aligned}$
\\\hline
\end{xltabular}
\vspace{0.5cm}
\end{document}